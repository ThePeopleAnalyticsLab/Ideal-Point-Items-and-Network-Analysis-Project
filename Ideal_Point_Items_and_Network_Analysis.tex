\documentclass[man]{apa6}
\usepackage{lmodern}
\usepackage{amssymb,amsmath}
\usepackage{ifxetex,ifluatex}
\usepackage{fixltx2e} % provides \textsubscript
\ifnum 0\ifxetex 1\fi\ifluatex 1\fi=0 % if pdftex
  \usepackage[T1]{fontenc}
  \usepackage[utf8]{inputenc}
\else % if luatex or xelatex
  \ifxetex
    \usepackage{mathspec}
  \else
    \usepackage{fontspec}
  \fi
  \defaultfontfeatures{Ligatures=TeX,Scale=MatchLowercase}
\fi
% use upquote if available, for straight quotes in verbatim environments
\IfFileExists{upquote.sty}{\usepackage{upquote}}{}
% use microtype if available
\IfFileExists{microtype.sty}{%
\usepackage{microtype}
\UseMicrotypeSet[protrusion]{basicmath} % disable protrusion for tt fonts
}{}
\usepackage{hyperref}
\hypersetup{unicode=true,
            pdftitle={Using Ideal Point Assessments to Develop Ideal Employees: An Inductive Approach},
            pdfauthor={Dan Simonet, Christopher M. Castille, \& Alexandra Harris},
            pdfborder={0 0 0},
            breaklinks=true}
\urlstyle{same}  % don't use monospace font for urls
\usepackage{graphicx,grffile}
\makeatletter
\def\maxwidth{\ifdim\Gin@nat@width>\linewidth\linewidth\else\Gin@nat@width\fi}
\def\maxheight{\ifdim\Gin@nat@height>\textheight\textheight\else\Gin@nat@height\fi}
\makeatother
% Scale images if necessary, so that they will not overflow the page
% margins by default, and it is still possible to overwrite the defaults
% using explicit options in \includegraphics[width, height, ...]{}
\setkeys{Gin}{width=\maxwidth,height=\maxheight,keepaspectratio}
\IfFileExists{parskip.sty}{%
\usepackage{parskip}
}{% else
\setlength{\parindent}{0pt}
\setlength{\parskip}{6pt plus 2pt minus 1pt}
}
\setlength{\emergencystretch}{3em}  % prevent overfull lines
\providecommand{\tightlist}{%
  \setlength{\itemsep}{0pt}\setlength{\parskip}{0pt}}
\setcounter{secnumdepth}{0}
% Redefines (sub)paragraphs to behave more like sections
\ifx\paragraph\undefined\else
\let\oldparagraph\paragraph
\renewcommand{\paragraph}[1]{\oldparagraph{#1}\mbox{}}
\fi
\ifx\subparagraph\undefined\else
\let\oldsubparagraph\subparagraph
\renewcommand{\subparagraph}[1]{\oldsubparagraph{#1}\mbox{}}
\fi

%%% Use protect on footnotes to avoid problems with footnotes in titles
\let\rmarkdownfootnote\footnote%
\def\footnote{\protect\rmarkdownfootnote}


  \title{Using Ideal Point Assessments to Develop Ideal Employees: An Inductive
Approach}
    \author{Dan Simonet\textsuperscript{1}, Christopher M.
Castille\textsuperscript{2}, \& Alexandra Harris\textsuperscript{3}}
    \date{}
  
\shorttitle{Developing Ideal Employees}
\affiliation{
\vspace{0.5cm}
\textsuperscript{1} Montclair State University\\\textsuperscript{2} Nicholls State University\\\textsuperscript{3} University of Georgia}
\usepackage{csquotes}
\usepackage{upgreek}
\captionsetup{font=singlespacing,justification=justified}

\usepackage{longtable}
\usepackage{lscape}
\usepackage{multirow}
\usepackage{tabularx}
\usepackage[flushleft]{threeparttable}
\usepackage{threeparttablex}

\newenvironment{lltable}{\begin{landscape}\begin{center}\begin{ThreePartTable}}{\end{ThreePartTable}\end{center}\end{landscape}}

\makeatletter
\newcommand\LastLTentrywidth{1em}
\newlength\longtablewidth
\setlength{\longtablewidth}{1in}
\newcommand{\getlongtablewidth}{\begingroup \ifcsname LT@\roman{LT@tables}\endcsname \global\longtablewidth=0pt \renewcommand{\LT@entry}[2]{\global\advance\longtablewidth by ##2\relax\gdef\LastLTentrywidth{##2}}\@nameuse{LT@\roman{LT@tables}} \fi \endgroup}


\DeclareDelayedFloatFlavor{ThreePartTable}{table}
\DeclareDelayedFloatFlavor{lltable}{table}
\DeclareDelayedFloatFlavor*{longtable}{table}
\makeatletter
\renewcommand{\efloat@iwrite}[1]{\immediate\expandafter\protected@write\csname efloat@post#1\endcsname{}}
\makeatother
\usepackage{setspace} \makeatletter
\renewcommand{\maketitle}{%
  \singlespacing{%
    \@ifundefined{@title}{}{%
      Poster\newline
      TITLE\newline
      \@title\par\par
    }
    
    \@ifundefined{@shorttitle}{}{%
      SHORTENED TITLE\newline
      \@shorttitle\par\par
    }
  
    \@ifundefined{@abstract}{}{%
      ABSTRACT\newline
      \@abstract\par\par
    }
    
    PRESS PARAGRAPH\newline
    In recent years interest in ideal point measures has increased in the field of personality assessment among both researchers and practitioners. This submission focuses on the utility of the application of ideal point analysis to gain a more accurate profile of both potential and current employees. More specifically, we focus on how network analysis, if applied to ideal point personality assessment data, might be used to coach employees into ideal organizational citizens. 


    \par\par
    
    WORD COUNT\newline
    2588
    \clearpage
  }
} \makeatother

\

\abstract{
We explore the possibility of applying network analysis to ideal point
personality assessment data. We illustrate how employee personality data
might be used as part of coaching interventions in order to develop
employee into ideal organizational citizens.
}

\begin{document}
\maketitle

\section{Introduction}\label{introduction}

Recent findings suggests self-report inventories conform to an ideal
point process in which persons endorse items only to the extent the item
content reflects the respondent's level of the attribute, or \(\Theta\).
Modeling the correct response process in personality assessment produces
several psychometric benefits, such as improved dimensionality, higher
total test information, and revelation of curvilinear effects (Drasgow,
Chernyshenko, \& Stark, 2010). One reason for such gains is the ideal
point perspective retains items assessing low, intermediate, and high
trait values and, thus, creates an instrument providing greater
measurement precisions across a broad range of a targeted attribute.

Due to its methodological nature, ideal-point research has focused
largely on technical issues, such as appropriate item writing
strategies, model fitting, or validity gains. Surprisingly little work
has been done to explore the advantages of ideal-point inventories to
understanding personality itself, such as explaining where traits come
from, how they operate, and how they produce differences in behavior.
These questions lie at the heart of the discipline (Fleeson \&
Jayawickreme, 2015) and carry theoretical implications for understanding
why personality predicts work behavior and how personality changes over
time (i.e., selection and development). Given that ideal-point
inventories capture a wider array of elemental differences in emotions,
thoughts, and behaviors constituting the Big Five, they may especially
suited to identifying plausible mechanisms through which personality
processes (deliberation, emotional regulation) accrue to form traits
(individual differences).

Drawing upon a psychometric network approach to individual differences
(A. O. J. Cramer et al., 2012), we recast the Big Five as a dynamic
system of directly interacting feelings, thoughts, and behaviors. Rather
than treat \enquote{hidden traits} as causal forces lying behind stable
behavioral patterns, the network approach models traits as consequences
of mutually reinforcing interactions between specific thoughts,
feelings, and behaviors (see Figure 1 for illustration). From this
perspective, discrete actions like working hard to attain long-term
goals, planning one's week, and focusing on a task to completion in a
person high on Conscientiousness do not co-occur because of a top-down
latent disposition, but because deciding to care about a long-term goal
leads one to be more disciplined in allocation of personal resources.
Forces bonding autonomous acts into trait clusters might be shared
biological origins, learning principles, socially enforced norms, or
functional aims that produce accretion of multiple explanatory
mechanisms which unite for causal, homeostatic, or logical reasons (A.
O. J. Cramer et al., 2012; Fleeson \& Jayawickreme, 2015; Wood, Gardner,
\& Harms, 2015).

\begin{figure}

{\centering \includegraphics[width=4.32in]{Latnet_fig1} 

}

\caption{Trait model according to a latent variable (left panel) and a network perspective (right panel)}\label{fig:unnamed-chunk-2}
\end{figure}

More importantly, the network perspective can provide a better view of
the cognitive, motivational, and functional dynamics characterizing the
development of the personality system, therefore favoring empirical
investigations of such mechanisms. Incorporating ideal point items may
offer further insight into trait development by pinpointing intermediate
ranges of a trait continuum (i.e., nodes) which incrementally
\enquote{bridge} personality components across distinct clusters
(Borsboom \& Cramer, 2013). For instance, the conscientiousness item
\enquote{I tend to be disorderly but also like to keep certain things
tidy} may bridge the agreeableness item of \enquote{I don't like to let
others down} to the remaining network of conscientiousness items. Why?
Because development in compassion arising from social roles (e.g.,
serious relationships, care for family) might elevate conscientiousness
by causing individuals to start bringing personal affairs in order. That
is, when we begin caring about others we may try to get our \enquote{act
together} in order to meet social responsibilities. Such effects may be
less evident in extreme items (I always keep my affairs in order)
because developmental processes are gradual and better seen in
intermediate steps. By finding and pulling these functional levers
(i.e., intermediate items), we may be able to nudge people to change in
productive ways on multiple dimensions (or traits) which has
implications for executive coaching and trait interventions.

The current study unifies these methodological innovations by applying
network analyses to a Big Five instrument developed with ideal-point
item writing strategies. We contrast four major network properties with
research exploring similar properties of common Big Five inventories
(Costantini \& Perugini, 2018; Costantini et al., 2015; A. O. J. Cramer
et al., 2012). The first is the topology, or \emph{large-scale
structure}, of the Big Five including global node arrangement and degree
to which nodes cluster together while distances between any two nodes
remain small (\emph{small-worldness}; Costantini et al., 2015). Two, we
identify the nature and content of cross-trait item pairings to identify
possible \emph{bridging} components explaining observed covariance
between trait factors (e.g., why do agreeable people tend to be
conscientious). Three, we compare the most \enquote{central} and
\enquote{peripheral} nodes with the nature of the central facets
identified in past publications. Nodes which are central play a more
prominent role in connecting elements of the personality system and,
consequentially, may be idea targets for intervention if desiring to
shift one's personality. Finally, given the general importance of
emotional stability and conscientiousness for job performance across
occupations, we examine the \emph{shortest} pathways that may explain
the route through which changes in emotional stability
(conscientiousness) may facilitate changes in conscientiousness
(emotional stability). In all cases, we highlight areas where
ideal-point items play a role in facilitating information flow in the
Big Five network.

\section{Methods and Results}\label{methods-and-results}

Given space limits yet novelty of network terminology, the methods and
results are presented concurrently. The data for this study come from a
sample (n = 677) of working employees from Amazon's Mechanical Turk
(MTurk) who completed a suite of ideal point personality assessments
capturing the Big Five Aspects (see Castille, 2015; DeYoung, 2015). The
marginal reliabilities of these assessments were all acceptable (ρ
\textgreater{} .82). Psychometric network analyses (Costantini et al.,
2015) involved casting items as nodes connected by lines, the strength
of which corresponds to the strength of the GLASSO regularized partial
correlations linking said nodes. The initial network is presented in
Figure 1 (item labels provided in Table 1).

\begin{verbatim}
## Warning: Missing column names filled in: 'X1' [1]
\end{verbatim}

Personality networks present items as nodes connected by edges
representing statistical relationships. We implemented a Gaussian
Graphical Model (GGM) on a polychoric correlation matrix using a
graphical least absolute shrinkage and selection (glasso) with the
extended EBIC criterium in \emph{qgraph} 1.4.3 (Epskamp, Borsboom, \&
Fried, 2018). There are two things to note. First, the glasso avoids
spurious associations by using \emph{regularization} to assign penalties
so all edges are shrunk with small edges being set to zero. This results
in a \emph{sparse} (i.e., conservative) network that safeguards against
overfitting by modeling covariance among components with as few
connections as possible. Second, because the network uses partial
correlations, all edges imply a relationship exists after controlling
for all other nodes. Because the model is uniquely specified, it
facilitates clear and unambiguous interpretation of edge-weight
parameters as the strength of \emph{unique} associations providing a
putative causal skeleton. Given the larger number of items, the EBIC
hyperparameter was set to a conservative .8 to err on the side of
caution and we hide all partial correlations less than .05 for visual
clarity.

The initial network presented in Figure 1 (item labels are provided in
Table 1) has 1,339 nonzero edges out of 11,175 possible edges (12\%).
Several insights can be inferred about the architecture and generating
processes of the Big Five. One, similar to (A. O. J. Cramer et al.,
2012), there is clustering for four of the Big Five with Openness
showing less cohesion. Two, it is possible to identify \enquote{pockets}
of high interactivity (i.e., facets or unique item effects) by
highlighting nodes with numerous, densely connected edges as well as
their pathways to the larger network. For instance, there is a
leadership pocket in the bottom of the extraversion network consisting
of items about taking charge (Ex33), following others (Ex31), or
enjoyment of project leadership (Ex34). Notice this cluster -- while
embedded in extraversion -- is also distinct because it is only
connected to the larger network by a few nodes, such as the belief one
is able to persuade (Ex35) and make friends (Ex15) coupled with efforts
to engage others (Ex5) and being averse to mediocre work (Co16). Its
distinction and peripheral placement may suggest assertive aspects of
Extraversion arise from social skills, effort to meet others, and a
desire to improve the status quo (i.e., not be mediocre). Three, items
along the Big Five borders may illuminate developmental pathways or
feedback loops through which change spreads between traits. Take the
conscientiousness and openness boundaries. The nodes in the lower left
suggest the enjoyment of solving complex problems (O2, O3, O8) is
positively linked to a high drive for achievement (Co17, Co18, Co19)
whereas nodes in the center left show tolerance for variety (O9, O10,
O11) as \emph{negatively} linked to preference for order (Co8, Co7, Co6,
Co9). Such countervailing effects suggest reinforcing gains in two
Openness components may be associated with diverging effects in a
person's Conscientiousness network (e.g., more industrious but lower
order). Interestingly, there are multiple boundary spanning items with
ideal-point properties (e.g., O9, Es16, Es27, Ex17, Ex18, Co2, Co8,
Co11, Co12, Ag7, Ag14, Ag21) suggesting they help elaborate unique ways
trait networks collide.

\begin{figure}
\centering
\includegraphics{Ideal_Point_Items_and_Network_Analysis_files/figure-latex/unnamed-chunk-4-1.pdf}
\caption{\label{fig:unnamed-chunk-4}Network representation of 168
ideal-point inventory modeled after the NEO-PI facet structure. Each
item is represented by a node, and the node number corresponds to the
item statements in Table 1. Nodes are connected by green (red) lines if
they are positively (negatively) correlated. Line thickness corresponds
to correlation strength. The spring-bsaed algorithm (Fruchterman \&
Reingold, 1991) used to generate the graph places strongly correlated
nodes closely together and towards the middle of the graph.}
\end{figure}

The \emph{small world} index was 2.35, which is higher than the values
of 1.01 reported on the HEXACO facets (Costantini et al., 2015) but
slightly lower than the 3 threshold recommended for describing a network
as a small-world (Watts \& Strogatz, 1998). When a network shows
small-worldness, changes in any random part of the network could quickly
spread across the whole system by allowing different clusters (e.g.,
traits) to directly influence one another (Watts \& Strogatz, 1998).
Results suggest the current inventory is more clearly organized than
dominance-based questionnaires into separate sub-systems (e.g., the Big
Five with exception of Openness) which themselves influence one another
by means of bridging connections. To illustrate the bridging components
linking the Big Five, the initial network was arranged by the Big Five
clusters with only partial correlations \textgreater{} .10 displayed
(see Figure 2). Eighteen cross-trait item pairings remained (presented
in Table 2) which might explain the often-substantial inter-correlations
observed between personality factors. Whereas the reason for some
linkages is not apparent (Ag23/Co3), others are commonly alluded to in
the literature such as the demand for both positive affect and difficult
goals in being ambitious and driven at work (e.g., Co15/Ex22). This
small-world structure may be masked in personality forms developed to
conform to simple structure (Constantini \& Perugi, 2016) suggesting an
ideal point inventory may offer a more realistic depiction of the
personality system.

\begin{table}[!h]

\caption{\label{tab:unnamed-chunk-5}Eighteen Bridging Item Pairs}
\centering
\resizebox{\linewidth}{!}{
\begin{tabular}[t]{lll}
\toprule
ItemLabels & FirstItem & SecondItem\\
\midrule
Ag28.O26 & People often tell me Im a genuine person & People talk to me because I empathize with how they feel\\
Ag13.Co5 & Honesty is the foundation of any good relationship & It is best to be careful when a decision has significant consequence\\
Ag14.Co25 & I feel the urge to confide in others & Although I am capable of motivating myself to complete tasks I prefer to have someone else prompting\\
Ag26.Co11 & Manipulating others can be helpful & I have lied to protect other people\\
Ag23.Co3 & Fine being anonymous when giving money to charity & On occasion it can be helpful to consider all options when making decisions\\
\addlinespace
Co5.O24 & It is best to be careful when a decision has significant consequences & If an emotion is really obvious then I can probably identify it\\
Co10.O9 & I like to plan my days in advance & I prefer stability or consistency to variety and change\\
Co18.O8 & I aspire to do well in more areas compared to most people & I really enjoy trying to tackle the most complex problems imaginable\\
Co3.O24 & On occasion it can be helpful to consider all options when making decisions & If an emotion is really obvious then I can probably identify it\\
Co15.Ex22 & I avoid setting goals but when I do I set extremely easy goals & I generally prefer activities that require little energy\\
\addlinespace
Ex9.Es18 & I always look at the bright side of life & I always feel great about the person that I am\\
Ex16.Es2 & I am always friendly to people & I like to consider myself as a very easygoing person\\
Ex22.O3 & I generally prefer activities that require little energy & I dislike thinking too hard about things\\
Ex12.O20 & I always hide my true feelings from people & I am unable to reciprocate when someone talks about their feelings\\
Co1.Es14 & I find that most all of my decisions are impulsive & I feel most alive when I give into my urges\\
\addlinespace
Co11.Es13 & I have lied to protect other people & Sometimes I do things I later regret\\
Ag25.Ex12 & I always hide my motives to get what I want & I always hide my true feelings from people\\
Ag3.Ex16 & When someone is in need I feel as though I have to help & I am always friendly to people\\
\bottomrule
\end{tabular}}
\end{table}

\begin{figure}
\centering
\includegraphics{Ideal_Point_Items_and_Network_Analysis_files/figure-latex/unnamed-chunk-6-1.pdf}
\caption{\label{fig:unnamed-chunk-6}Same network results from Figure 1
rearranged by Big Five groupings and restricted to display partial
correlations .10 or greater. Visualized edges depict strong residual
item associations within and between trait factors.}
\end{figure}

\subsubsection{Centrality Estimation}\label{centrality-estimation}

A typical way of assessing node importance is to compute centrality
indices of the network structure (Costantini et al., 2015; Newman, 2010;
Opsahl, Agneessens, \& Skvoretz, 2010). Three such measures are (1)
\emph{node strength}, quantifying how well a node is directly connected
to other nodes by summing all of its absolute edges, (2)
\emph{closeness}, quantifying how well a node is indirectly connected to
other nodes by taking the inverse of all shortest path lengths between
the node and all other nodes, and (3) \emph{betweeness}, quantifying how
important a node is in the average path between two other nodes. While
such indices often agree, it is possible for a node to be high on one
index but low on another. For instance, the Amsterdam airport would
score high on \emph{strength} as many airports fly planes in and out of
Amsterdam. Comparatively, the airport in Anchorage, Alaska, while low on
strength in terms of absolute number of connections, is actually higher
than Amsterdam on \emph{betweenness} because it serves as a common hub
indirectly connecting many international airports to each other via
oversea flights.

The centrality plots appear in Figure 4. Several Agreeableness and
Conscientiousness items were highly influential across indices,
especially those dealing with manipulation (Ag25, Ag26), deliberation in
action and decision making (Co1, Co5, Co20), or holding
\enquote{moderate} amounts of motivation (Co31, Co25). The most central
conscientiousness items reflect both the \enquote{inhibitive} pole of
the trait, recognized in facets broadly referring to control over one's
impulses as seen in facets such as \enquote{orderly} (Jackson et al.,
2010) or \enquote{self-control} (Roberts, Chernyshenko, Stark, \&
Goldberg, 2005), and \enquote{modest} levels of the \enquote{proactive}
pole, reflected in ideal-point versions of facets labeled
\enquote{achievement striving} (Costa McCrae, \& Dye, 1991) or
\enquote{industriousness} (Roberts et al., 2005). Similar to past
network analyses (Costantini et al., 2015), changes in inhibitory
tendencies are more likely to influence the wider personality network
(most likely through fringe elements of conscientiousness) whereas
changes in other portions of the personality network would similarly
impact tendencies towards restraint. More interesting, the proactive
ideal-point conscientiousness items (Co31, Co25) had higher centrality
indices due to their role in linking the lager conscientiousness network
to agreeableness and extraversion.

Items from additional traits also had relatively high
betweeness-centrality, meaning they occupied strategic positions
connecting several groups of nodes that would be connected by longer
paths without these particular items. These include Ex9 (I always look
at the bright side of life), Ex16 (I am always friendly to people), Es18
(I always feel great about the person that I am), Es11 (I have a good
amount of control on my cravings), and, to a lesser extent, O7 (I enjoy
having abstract or philosophical conversations). By examining Figure 1
you can visualize in what respect these nodes serve as important
mediators in connecting items. For instance, E9's focus on optimism
helps bridge multiple components of extraversion with fear and
self-evaluative components of emotional stability (Es28, Es27, Es18).

\begin{figure}
\centering
\includegraphics{Ideal_Point_Items_and_Network_Analysis_files/figure-latex/unnamed-chunk-7-1.pdf}
\caption{\label{fig:unnamed-chunk-7}Centrality plot depicting the
betweenness, closeness, and strength of each node.}
\end{figure}

\subsubsection{Shortest Pathway between Emotional Stability and
Conscientiousness}\label{shortest-pathway-between-emotional-stability-and-conscientiousness}

Finally, a network illustrating the shortest paths between all
conscientiousness and emotional stability items was computed (see Figure
5). In comparison to the first network, these networks clarify possible
pathways and mediating items between these two factors. The shortest
path between 2 nodes represents the minimum number of steps needed to go
from one node to another, and is computed using Dijkstra's algorithm
(Dijkstra, 1959). This can be seen as a roadmap including all possible
routes from destination A to destination B, but only one of these routes
being quicker---this would then be the route highlighted in the shortest
path network.

Our network illustrates the shortest path between multiple items hence
highlights a diverse array of routes linking conscientiousness and
emotional stability. A few general observations. First, the nodes Co21
(Tendency to misjudge situations), Es13 (Sometimes do things I later
regret), and, more indirectly, Es18 (Always feel great about person I
am), Es14 (Feel most alive when giving into urges), and Ex16 (I am
always friendly to people) are primary hubs for multiple pathway which
indirectly link both item sets. Interestingly, several of these
intermediate items share a self-reflective, guilt-laden connection, such
that taking time to correctly assess the consequences of one's decision
lessens the likelihood of impulsively engaging in actions which lead to
remorse and low self-esteem. It may be possible the links between
regulation of emotions and motivation can be explained by a realization
hasty actions lead to bad consequences. On a more global level, whereas
most of the emotional stability items clustered together to flow into
conscientiousness, the more diffuse conscientiousness network flowed
down into emotional stability through a few, primarily ideal-point
oriented behaviors of Co24 (Pride myself on unwavering ability to act
responsibly), Co25 (Although capable of self-motivation, I prefer to
have someone else provide direction), Co31 (Do just enough work to get
by), and Co11 (I have lied to protect others). In other words, there
appear to be many routes for emotional stability change to affect
conscientiousness but only a few primary routes (primarily in being
responsible or industrious) for conscientiousness to spread into
emotional stability.

\begin{figure}
\centering
\includegraphics{Ideal_Point_Items_and_Network_Analysis_files/figure-latex/unnamed-chunk-8-1.pdf}
\caption{\label{fig:unnamed-chunk-8}Network depicting the shortest paths
between Conscientiousness and Emotional Stability items. Edges belonging
to the shortest-paths are full, while the other edges are dashed.}
\end{figure}

\section{Conclusion}\label{conclusion}

Our results suggests that there is promise in exploring the
developmental applications of ideal point inventories. Many plausible
bridges exist linking developmental trajectories across the personality
system. Future research investigating these bridges is needed.
Importantly for this symposium, the results indicate personality
networks assessed by ideal point inventories are more clearly organized
than dominance-based questionnaires (see Constantini et al., 2015),
further bolstering the notion ideal point assessments offer a more
realistic depiction of personality. Perhaps we will find that ideal
point assessments can help practitioners develop ideal employees.

\newpage

\section{References}\label{references}

\setlength{\parindent}{-0.5in} \setlength{\leftskip}{0.5in}

\begin{verbatim}
## Warning in styling_latex_scale_down(out, table_info): Longtable cannot be
## resized.
\end{verbatim}

\begin{longtable}{ll}
\toprule
Labels & Items\\
\midrule
\endfirsthead
\multicolumn{2}{@{}l}{\textit{(continued)}}\\
\toprule
Labels & Items\\
\midrule
\endhead
\
\endfoot
\bottomrule
\endlastfoot
O1 & I find theoretical conversations extremely boring\\
O2 & I dislike focusing on difficult problems\\
O3 & I dislike thinking too hard about things\\
O4 & I prefer to focus on mentally stimulating projects  but sometimes it is nice to have time to mentally relax\\
O5 & Sometimes I enjoy solving complex problems\\
\addlinespace
O6 & I enjoy solving complex problems\\
O7 & I enjoy having abstract or philosophical conversations\\
O8 & I really enjoy trying to tackle the most complex problems imaginable\\
O9 & I prefer stability or consistency to variety and change\\
O10 & I like change  but I also need stability\\
\addlinespace
O11 & While I do somewhat prefer variety  I also enjoy stability or consistency\\
O12 & I find all artwork to be similar\\
O13 & Listening to poetry or music seems to be a waste of time\\
O14 & While listening to music is nice  it is pointless\\
O15 & From time to time I like to appreciate the beauty around me\\
\addlinespace
O16 & There have been times when a song has made me emotional\\
O17 & I see some value in art and beauty\\
O18 & I like to think about real world problems\\
O19 & People have told me I am emotionally inept\\
O20 & I am unable to reciprocate when someone talks about their feelings\\
\addlinespace
O21 & It takes me a long time to understand other people s emotions\\
O22 & Unless someone tells me how they feel I won t know for sure\\
O23 & I sometimes can tell how people feel\\
O24 & If an emotion is really obvious then I can probably identify it\\
O25 & For the most part I understand others  emotions\\
\addlinespace
O26 & People talk to me  because I can empathize with how they feel\\
O27 & I have a deep understanding of others  emotions\\
Es1 & I am rarely frustrated by anything\\
Es2 & I like to consider myself as a very easygoing person\\
Es3 & I rarely get irritated by others\\
\addlinespace
Es4 & I am somewhat balanced in my experience of anger\\
Es5 & I am somewhat balanced in my experience of frustration\\
Es6 & I get angry easily\\
Es7 & I get frustrated easily\\
Es8 & I have a very short temper\\
\addlinespace
Es9 & I often resist my temptations\\
Es10 & People say I have great  self control\\
Es11 & I have a good amount of control on my cravings\\
Es12 & I indulge reasonably when I feel inclined to do so\\
Es13 & Sometimes I do things I later regret\\
\addlinespace
Es14 & I feel most alive when I give into my urges\\
Es15 & I rarely get stressed out about things\\
Es16 & Sometimes I get caught up in my problems  and other times I try not to worry about things that have already happened\\
Es17 & I get caught up in my problems\\
Es18 & I always feel great about the person that I am\\
\addlinespace
Es19 & I seldom feel down in the dumps\\
Es20 & On occasion  I feel blue  but most of the time I don t feel blue\\
Es21 & My mood changes about half the time\\
Es22 & My mood changes all the time\\
Es23 & I rarely become embarrassed\\
\addlinespace
Es24 & I am always extremely afraid that I will do the wrong thing\\
Es25 & I rarely panic\\
Es26 & Occasionally I panic  but I usually do not\\
Es27 & Sometimes I panic easily  and other times I do not\\
Es28 & My emotions usually get the best of me\\
\addlinespace
Ex1 & I am a socially awkward person\\
Ex2 & I sometimes feel uncomfortable when surrounded by a big crowd\\
Ex3 & I prefer to socialize in small groups\\
Ex4 & I like to do most things in large groups\\
Ex5 & I constantly try to engage with different people\\
\addlinespace
Ex6 & People often refer to me as a  downer\\
Ex7 & I am somewhat of a fun person to be around\\
Ex8 & I like to focus on the positive side of things\\
Ex9 & I always look at the bright side of life\\
Ex10 & I am an incredibly joyful person to be around\\
\addlinespace
Ex11 & I am incredibly uptight around others\\
Ex12 & I always hide my true feelings from people\\
Ex13 & I usually find it hard to make friends\\
Ex14 & I am usually quiet when I meet new people\\
Ex15 & I usually find it easy to make friends\\
\addlinespace
Ex16 & I am always friendly to people\\
Ex17 & I don t mind loud parties  but I don t prefer them either\\
Ex18 & I tend to seek adventure\\
Ex19 & Loud parties can definitely be fun\\
Ex20 & I couldn t live without adventure\\
\addlinespace
Ex21 & I always take my time   even when a faster pace may be needed\\
Ex22 & I generally prefer activities that require little energy\\
Ex23 & Half of the time I prefer leisurely activities and half of the time I prefer activities to be fast paced\\
Ex24 & Compared to extremely energetic people  I am somewhat less energetic\\
Ex25 & My fast paced lifestyle keeps me more busy than most\\
\addlinespace
Ex26 & My lifestyle requires a high energy level\\
Ex27 & I always try to live life to the fullest extent that I possibly can\\
Ex28 & Compared to most people  I live a very fast paced life\\
Ex29 & I hate leading groups\\
Ex30 & I have no interest in leadership\\
\addlinespace
Ex31 & I would rather follow directions than lead\\
Ex32 & From time to time  I enjoy taking charge on projects  but some other times I prefer others to take the lead\\
Ex33 & I am often the person to take charge of a group\\
Ex34 & I enjoy taking the lead on new projects\\
Ex35 & I can always persuade people to follow my lead\\
\addlinespace
Ex36 & I always end up leading the groups I participate in\\
Co1 & I find that most all of my decisions are impulsive\\
Co2 & I sometimes make decisions based on instinct rather than facts  and sometimes I prefer facts\\
Co3 & On occasion it can be helpful to consider all options when making decisions\\
Co4 & I prefer to have backup plans\\
\addlinespace
Co5 & It is best to be careful when a decision has significant consequences\\
Co6 & I have difficulties working on a clean and organized desk\\
Co7 & Organization is not a priority for me\\
Co8 & While I like order and regularity  I also enjoy when things are a bit chaotic\\
Co9 & I keep my workstation somewhat clean and tidy\\
\addlinespace
Co10 & I like to plan my days in advance\\
Co11 & I have lied to protect other people\\
Co12 & I aim to tell the truth as often as possible  but I can think of numerous situations that have required me to bend the truth\\
Co13 & I try to keep all of the promises I make  but sometimes I am unable to deliver on them\\
Co14 & Regardless of the situation  I always tell the truth\\
\addlinespace
Co15 & I avoid setting goals  but when I do  I set extremely easy goals\\
Co16 & I am fine being an average worker\\
Co17 & I have a drive to succeed in my work\\
Co18 & I aspire to do well in more areas compared to most people\\
Co19 & I work extremely hard to be the very best at everything I do\\
\addlinespace
Co20 & I put little thought into my actions\\
Co21 & I have a tendency to misjudge situations\\
Co22 & I tend to perform in most areas at the average level of other people\\
Co23 & While I often excel in what I do  I also have much to learn to be better\\
Co24 & I pride myself on my unwavering ability to act responsibly\\
\addlinespace
Co25 & Although I am capable of motivating myself to complete tasks  I prefer to have someone else prompting me\\
Co26 & More often than not  I depend on myself rather than others for the motivation needed to successfully complete a task\\
Co27 & Even when tasks are difficult  I find a way to complete them\\
Co28 & I always get my work in on time\\
Co31 & I do just enough work to get by\\
\addlinespace
Co32 & I find it difficult to start my work\\
Co33 & I prefer making decisions quickly rather than after thoroughly thinking things through\\
Ag1 & Being a winner is much more important than being cooperative\\
Ag2 & Cooperating with others is equally as important as winning\\
Ag3 & When someone is in need  I feel as though I have to help\\
\addlinespace
Ag4 & Cooperating with others is more important than winning\\
Ag5 & I always put the needs of others before my own\\
Ag6 & I am extremely self centered\\
Ag7 & I sometimes help a friend because it s the right thing to do  other times is because I want something in return\\
Ag8 & I frequently think about how others are doing\\
\addlinespace
Ag9 & I worry about how people are doing\\
Ag10 & I live to serve others\\
Ag11 & Everyone has hidden intentions\\
Ag12 & I find it easier to trust in some people than in others\\
Ag13 & Honesty is the foundation of any good relationship\\
\addlinespace
Ag14 & I feel the urge to confide in others\\
Ag15 & If someone wrongs me  it is difficult for me to forgive them\\
Ag16 & Sometimes I am easy to satisfy  but other times I can seem a bit pushy\\
Ag17 & While I sometimes forgive others to avoid confrontation  I also often challenge others\\
Ag18 & People who know me would likely say I am generally a forgiving person\\
\addlinespace
Ag19 & I usually try to satisfy others  needs  rather than my own when I sense conflict emerging\\
Ag20 & People who know me would say I am an extremely forgiving person\\
Ag21 & I shy away from credit sometimes  but other times it is nice to be recognized\\
Ag22 & Sometimes the work I do is really excellent  other times it is mediocre\\
Ag23 & When I give money to a charity  I am fine with being anonymous\\
\addlinespace
Ag24 & I always share the credit I receive on teamwork\\
Ag25 & I always hide my motives to get what I want\\
Ag26 & Manipulating others can be helpful\\
Ag27 & I use flattery on occasion when dealing with others\\
Ag28 & People often tell me that I am a genuine person\\*
\end{longtable}

\hypertarget{refs}{}
\hypertarget{ref-borsboomNetworkAnalysisIntegrative2013}{}
Borsboom, D., \& Cramer, A. O. (2013). Network Analysis: An Integrative
Approach to the Structure of Psychopathology. \emph{Annual Review of
Clinical Psychology}, \emph{9}(1), 91--121.
doi:\href{https://doi.org/10.1146/annurev-clinpsy-050212-185608}{10.1146/annurev-clinpsy-050212-185608}

\hypertarget{ref-castilleBrightDarkVirtues2015}{}
Castille, C. M. (2015). \emph{Bright or dark, or virtues and vices? A
reexamination of the big five and job performanc}. Unpublished Doctoral
Dissertation, Louisiana Tech University.

\hypertarget{ref-costantiniFrameworkTestingCausality2018}{}
Costantini, G., \& Perugini, M. (2018). A Framework for Testing
Causality in Personality Research. \emph{European Journal of
Personality}, \emph{32}(3), 254--268.
doi:\href{https://doi.org/10.1002/per.2150}{10.1002/per.2150}

\hypertarget{ref-costantiniStateARtPersonality2015}{}
Costantini, G., Epskamp, S., Borsboom, D., Perugini, M., Mõttus, R.,
Waldorp, L. J., \& Cramer, A. O. J. (2015). State of the aRt personality
research: A tutorial on network analysis of personality data in R.
\emph{Journal of Research in Personality}, \emph{54}, 13--29.
doi:\href{https://doi.org/10.1016/j.jrp.2014.07.003}{10.1016/j.jrp.2014.07.003}

\hypertarget{ref-cramerDimensionsNormalPersonality2012}{}
Cramer, A. O. J., van der Sluis, S., Noordhof, A., Wichers, M.,
Geschwind, N., Aggen, S. H., \ldots{} Borsboom, D. (2012). Dimensions of
Normal Personality as Networks in Search of Equilibrium: You Can't Like
Parties if You Don't Like People: Dimensions of normal personality as
networks. \emph{European Journal of Personality}, \emph{26}(4),
414--431. doi:\href{https://doi.org/10.1002/per.1866}{10.1002/per.1866}

\hypertarget{ref-deyoungCyberneticBigFive2015}{}
DeYoung, C. G. (2015). Cybernetic Big Five Theory. \emph{Journal of
Research in Personality}, \emph{56}, 33--58.
doi:\href{https://doi.org/10.1016/j.jrp.2014.07.004}{10.1016/j.jrp.2014.07.004}

\hypertarget{ref-dijkstraNoteTwoProblems1959}{}
Dijkstra, E. W. (1959). A note on two problems in connexion with graphs.
\emph{Numerische Mathematik}, \emph{1}, 269--271.

\hypertarget{ref-drasgow75YearsLikert2010}{}
Drasgow, F., Chernyshenko, O. S., \& Stark, S. (2010). 75 Years After
Likert: Thurstone Was Right! \emph{Industrial and Organizational
Psychology}, \emph{3}(04), 465--476.
doi:\href{https://doi.org/10.1111/j.1754-9434.2010.01273.x}{10.1111/j.1754-9434.2010.01273.x}

\hypertarget{ref-epskampEstimatingPsychologicalNetworks2018}{}
Epskamp, S., Borsboom, D., \& Fried, E. I. (2018). Estimating
psychological networks and their accuracy: A tutorial paper.
\emph{Behavior Research Methods}, \emph{50}(1), 195--212.
doi:\href{https://doi.org/10.3758/s13428-017-0862-1}{10.3758/s13428-017-0862-1}

\hypertarget{ref-fleesonWholeTraitTheory2015}{}
Fleeson, W., \& Jayawickreme, E. (2015). Whole Trait Theory.
\emph{Journal of Research in Personality}, \emph{56}, 82--92.
doi:\href{https://doi.org/10.1016/j.jrp.2014.10.009}{10.1016/j.jrp.2014.10.009}

\hypertarget{ref-newmanNetworks2010}{}
Newman, M. (2010). \emph{Networks} (Second Edition.). Oxford, UK: Oxford
University.

\hypertarget{ref-opsahlNodeCentralityWeighted2010}{}
Opsahl, T., Agneessens, F., \& Skvoretz, J. (2010). Node centrality in
weighted networks: Generalizing degree and shortest paths. \emph{Social
Networks}, \emph{32}(3), 245--251.
doi:\href{https://doi.org/10.1016/j.socnet.2010.03.006}{10.1016/j.socnet.2010.03.006}

\hypertarget{ref-robertsStructureConscientiousnessEmpirical2005}{}
Roberts, B. W., Chernyshenko, O. S., Stark, S., \& Goldberg, L. R.
(2005). The structure of conscientiousness: An empirical investigation
based on seven major personality questionnaires. \emph{Personnel
Psychology}, \emph{58}(1), 103--139.
doi:\href{https://doi.org/10.1111/j.1744-6570.2005.00301.x}{10.1111/j.1744-6570.2005.00301.x}

\hypertarget{ref-wattsCollectiveDynamicsSmallWorld1998}{}
Watts, D., \& Strogatz, S. (1998). Collective Dynamics of Small-World
Networks. \emph{Nature}, \emph{393}, 3.

\hypertarget{ref-woodHowFunctionalistProcess2015}{}
Wood, D., Gardner, M. H., \& Harms, P. D. (2015). How functionalist and
process approaches to behavior can explain trait covariation.
\emph{Psychological Review}, \emph{122}(1), 84--111.
doi:\href{https://doi.org/10.1037/a0038423}{10.1037/a0038423}


\end{document}
